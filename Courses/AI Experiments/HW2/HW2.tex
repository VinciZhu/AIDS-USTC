\documentclass[12pt,a4paper]{article}

\input{math.tex}

\usepackage{ctex}
\usepackage{natbib}
\usepackage{siunitx}
\usepackage{footnote}
\usepackage{booktabs}
\usepackage{titletoc}
\usepackage{titlesec}
\usepackage{lastpage}
\usepackage{fancyhdr}
\usepackage{hyperref}
\usepackage{graphicx}
\usepackage{multirow}
\usepackage{enumitem}
\usepackage{subcaption}
\usepackage{tabularray}
\usepackage[left=3cm, right=3cm, top=3cm, bottom=3cm]{geometry}
\UseTblrLibrary{booktabs}
\UseTblrLibrary{siunitx}

\setCJKmainfont{SimSun}

\pagestyle{fancy}
\fancyhf{}
\renewcommand{\headrulewidth}{0pt}
\fancyfoot[C]{\thepage/\pageref{LastPage}}

\hypersetup{
    colorlinks=true,
    linkcolor=blue,
    filecolor=magenta,      
    urlcolor=cyan,
}

\begin{document}

{
\centering
\vspace*{3cm}
{\Huge\heiti\bfseries 中国科学技术大学人工智能实践课程\\技术报告\par}
\vspace{5cm}
{\Large 基于 BERT 和 RoBERTa 的中文文本二分类\\ 朱云沁 PB20061372\par}
\vfill
{\Large\fangsong 中国科学技术大学\\ 2024 年 1 月\par}
}
\newpage

\titlecontents{section}[1.5em]{\vspace{0.5em}\fangsong}{\contentslabel{1.5em}}{}{\hspace{0.5em}\titlerule*[4pt]{.}\contentspage}
\titlecontents{subsection}[5em]{\vspace{0.5em}\fangsong}{\contentslabel{2.5em}}{}{\hspace{0.5em}\titlerule*[4pt]{.}\contentspage}
\renewcommand{\contentsname}{\centering\heiti\bfseries 目录}
\tableofcontents
\newpage

\titleformat{\section}{\large\fangsong}{\thesection}{1em}{}
\titleformat{\subsection}{\fangsong}{\thesubsection}{1em}{}

\section{实验目标}

训练任务为文本分类. 数据集是标签为“0”和“1”的两类文本. 在样本不足的情况下完成对 Transformer 模型的训练, 尽可能正确分类文本.

\subsection{数据描述}

训练集中含有 1599 条文本, 其中 799 条标签为“0”, 800 条标签为“1”. 测试集中含有 401 条文本, 其中 199 条标签为“0”, 202 条标签为“1”. 每条文本的长度不一, 最长的文本含有 26 个字符. 我们将测试集进一步划分为 5 折, 用于交叉验证. 各子集中标签数量如下表所示.


\begin{table}[htbp]
    \centering\small
    \begin{tblr}{ccccccc}
        \toprule
                & \SetCell[c=5]{} Training Set &        &        &        &        & \SetCell[r=2]{} Test Set \\\cline{2-6}
                & Fold 1                       & Fold 2 & Fold 3 & Fold 4 & Fold 5 &                          \\
        \midrule
        \# of 0 & 160                          & 160    & 160    & 160    & 159    & 199                      \\
        \# of 1 & 160                          & 160    & 160    & 160    & 160    & 202                      \\
        \bottomrule
    \end{tblr}
    \caption{数据集划分}
\end{table}

通过随机采样, 得到训练集中若干文本样本如下.

\begin{table}[htbp]
    \centering\small
    \begin{tblr}{cl}
        \toprule
        Label & Text                  \\
        \midrule
        0     & 高考理科录取人数最多20个主流大众专业解析 \\
        0     & 中央民族大学2009年普通本科招生章程   \\
        0     & 研究生考试迟到15分钟不得入场       \\
        \midrule
        1     & 教育部就中小学教师队伍补充等有关工作答问  \\
        1     & 安理会半数成员反对巴勒斯坦加入联合国    \\
        1     & 美媒称国际刑事法庭决定逮捕苏丹领导人    \\
        \bottomrule
    \end{tblr}
    \caption{训练集中文本样本}
\end{table}

初步推测, 数据集可能来源于新闻标题文本, 标签“0”可能与高等教育、考试等话题相关, 标签“1”可能与国内政治、国际关系等话题相关.

\section{国内外研究现状}

文本分类是自然语言处理领域中的经典问题. 由于非结构化的特性, 从文本中提取信息极具挑战性. 传统方法利用一组预定义的规则将文本分类到不同的类别中, 并需要特定领域大量的专家知识, 目前已基本淘汰. 另一方面, 得益于神经网络强大的表达能力, 深度学习已经成为文本分类的主流方法.

\subsection{词元级别的表示学习}

用于文本分类的神经网络的核心组件是一个词嵌入模型, 将词元映射到低维连续特征向量. 这类模型通常采用自监督学习的方法, 在大规模文本语料上进行训练. 1989 年, Dumais 等人开发了潜在语义分析 (LSA) \cite{dumais_latent_2004}, 成为最早的词嵌入模型之一. 2001 年, Bengio 等人提出利用前馈神经网络学习语言的概率分布 \cite{bengio_neural_2003}. 2013 年, Google 研究团队提出 Skip-Gram \cite{mikolov_distributed_2013} 和 CBoW \cite{mikolov_efficient_2013}, 所得 word2vec 词嵌入在下游任务中得到广泛应用. 2014 年, Pennington 等人提出 GloVe \cite{pennington_glove_2014}, 通过统计全局词汇的共现信息来生成词向量. 2016 年, Bojanowski 等人提出 fastText \cite{bojanowski_enriching_2017}, 在具有相似结构的词之间共享来自子词的参数.

近年来, 上下文敏感的词表示学习得到了广泛关注. 2017 年, Peters 等人开发了一个基于双向 LSTM 的词嵌入模型 ELMo \cite{peters_deep_2018}, 由于捕捉了上下文信息, 相较 word2vec 效果显著提升. 与此同时, 随着 Transformer \cite{vaswani_attention_2017} 的提出, 词嵌入模型迎来了新一轮革命. 2018 年, OpenAI 开始使用 Transformer 构建语言模型, 提出的 GPT \cite{radford_improving_2018} 及后续版本已被广泛用于文本生成任务. 同年, Google 开发了基于双向 Transformer 编码器的 BERT \cite{devlin_bert_2019}. BERT-large 包含 3 亿个参数, 训练数据包括 33 亿个单词, 是当前最先进的词嵌入模型之一.

目前, 基于 Transformer 的词嵌入模型仍在不断发展. 代表性工作有 RoBERTa \cite{liu_roberta_2019}, XLNet \cite{yang_xlnet_2020}, ALBERT \cite{lan_albert_2020}, ELECTRA \cite{clark_electra_2020} 等. 这些模型在预训练阶段采用了改进的训练目标或模型架构, 在训练速度、模型大小、计算量等方面进行了优化, 在下游任务上取得了更好的效果. 中文预训练词嵌入模型的代表性工作包括: (1) 清华大学的 ERNIE \cite{zhang_ernie_2019}, 百度的 ERNIE 2.0 \cite{sun_ernie_2019} \& 3.0 \cite{sun_ernie_2021}, 核心思想在于通过知识增强预训练效果. (2) 哈工大讯飞联合实验室的 BERT-wwm \& RoBERTa-wwm \& MacBERT \cite{cui_revisiting_2020,cui_pre-training_2021}, 核心思想在于掩蔽中文整词以及采用更先进的预训练任务.

\subsection{序列级别的表示学习}

为了从词元的向量表示中提取对文本分类有用的信息, 通常的做法是设计专用的网络模块, 组合得到整个序列的向量表示. 2014 年, Le 和 Mikolov 提出 doc2vec \cite{le_distributed_2014}, 将各个词向量与段落向量进行平均或拼接, 用于文本分类. 同年, Kim 在预训练 word2vec 的基础上设计卷积神经网络, 用于文本分类 \cite{kim_convolutional_2014}. 循环神经网络, 例如 LSTM, 也被广泛地应用于提取序列表示, 代表性工作包括 \cite{cheng_long_2016}.

基于 Transformer 的预训练语言模型使得任务不可知的文本表示成为可能. 利用 BERT 等模型得到的上下文敏感的词向量, 通常只需简易的平均池化和全连接层即可较好地提取序列信息, 迁移至文本分类等下游任务, 并取得最先进的性能 \cite{devlin_bert_2019}. 针对特定任务设计卷积或循环神经网络不再必需. 因此, 预训练-微调已经成为当前文本分类的主流范式.


\section{课题总体方案设计}

根据调研结果, 我们选用 BERT 作为本实验的词嵌入模型, 因为其有着良好的生态、先进的性能和广泛的应用. 作为改进, 我们同样尝试 RoBERTa, 以期进一步提升模型性能. 本节将简要介绍基于 BERT 的模型架构, BERT 和 RoBERTa 的预训练过程, 以及文本分类器的设计.

\subsection{Transformer 文本分类网络}

基于 BERT 的文本分类模型架构如图 \ref{fig:bert-architecture} 所示.

\begin{figure}[htbp]
    \centering
    \includegraphics[width=0.8\textwidth]{./figs/bert.pdf}
    \caption{基于 BERT 的文本分类网络}\label{fig:bert-architecture}
\end{figure}

BERT 是一种基于 Transformer 的仅编码器的双向语言模型. 该模型首先将输入文本通过 WordPiece 分词器分割成词元序列, 并在首尾分别添加特殊的标记符号. 词元序列经过嵌入层, 得到每个词元的向量表示, 然后输入 Transformer 编码器. 编码器模块由残差连接、归一化、多头自注意力、逐位置前馈网络等部分组成. 其中, 自注意力机制在序列建模中起到了至关重要的作用. 其核心公式为

\begin{equation}
    \mathrm{Attention}(Q, K, V) = \mathrm{softmax}\left(\frac{QK^T}{\sqrt{d_k}}\right)
\end{equation}

其中, $Q$, $K$, $V$ 分别为查询、键、值的矩阵表示, $d_k$ 为键的维度. 输入序列中第 $i$ 个词元的向量表示 $\vx_i$ 与三个权重矩阵依次相乘, 得到第 $i$ 个查询 $\vq_i = \vx_i W_Q$, 键 $\vk_i = \vx_i W_K$, 值 $\vv_i = \vx_i W_V$. 通过归一化的点积计算注意力权重, 并与值矩阵相乘, 得到第 $i$ 个词元的输出 $\vy_i = \mathrm{Attention}(\vq_i, \vk_i, \vv_i)$. 一套权重矩阵 ($W_Q$, $W_K$, $W_V$) 称作一个注意力头, 多个注意力头并行计算, 并将结果拼接, 得到最终的输出 $\vy_i = \mathrm{Concat}(\vy_i^1, \vy_i^2, \dots, \vy_i^H) W_O$, 其中 $H$ 为注意力头的数量, $W_O$ 为输出矩阵. 通常, 设置 $d_k = d_v = d / H$, 其中 $d$ 为词向量的特征维度.

经多头自注意力混合后的词向量将输入到逐位置的前馈网络, 捕捉同一向量内不同特征维度之间的依赖关系. 逐位置前馈网络由两层全连接层组成, 两层之间有 GELU 激活函数, 隐层神经元数量为特征维度的 4 倍. 为了训练的稳定性, 在每一层的输入和输出之间都有残差连接和归一化操作. 一个基础尺寸的 BERT 模型 (即 BERT-base) 通常设置 Transformer 编码器的模块数为 $L = 12$, 注意力头的数量为 $H = 12$, 词向量的特征维度为 $d = 768$, 模型参数总量约为 1.1 亿.

得益于先进的序列建模方式和充分的自监督预训练, BERT 模型最终输出的词向量具有上下文敏感、任务不可知的优点, 只需极小程度的扩展即可用于文本分类的下游任务. 具体而言, 我们首先将序列中的所有词向量进行平均池化, 得到整个序列的向量表示. 然后, 通过一个双层感知机将序列向量映射到二分类的概率分布.

\subsection{预训练的改进: 从 BERT 到 RoBERTa}

本实验中, 我们直接加载 Google 提供的 \texttt{bert-base-chinese}\footnote{\url{https://huggingface.co/bert-base-chinese}} 和哈工大讯飞联合实验室提供的 \texttt{hfl/chinese-roberta-wwm-ext}\footnote{\url{https://huggingface.co/hfl/chinese-roberta-wwm-ext}} 预训练 BERT 权重. 出于完整性考虑, 我们仍简要介绍 BERT 和 RoBERTa 的预训练过程.

BERT 的预训练过程包含两个任务: (1) Masked Language Model (MLM), (2) Next Sentence Prediction (NSP). MLM 任务的目标是预测序列中被掩蔽的词元, 以此鼓励模型学习上下文信息. 例如, 对于输入句子“中科大真是牛校出牛子”, 我们随机选择一个单词, 将其替换为特殊的 [MASK] 词元, 得到“中科[MASK]真是牛校出牛子”, 并让模型预测被掩蔽的词元; NSP 任务的目标是预测两个句子是否连续, 以此鼓励模型学习句子之间的关系. 例如, 模型输入为“[CLS]中科大真是牛校出牛子[SEP]我也想去中科大[SEP]”, 则模型应当判断两个句子连续. 其中, [CLS] 和 [SEP] 为特殊的词元, 前者用于 NSP 分类任务, 后者用于分割句子. 通常, 以 50\% 的概率随机选择两个连续句子或任意句子作为输入对. 本实验中使用的 BERT-base 特指在中文维基百科上预训练得到的模型.

RoBERTa 同样采用 BERT 架构, 但对预训练过程进行了改进, 主要包括: (1) 采用动态掩码而非在预处理阶段进行的静态掩码. (2) 放弃了 NSP 训练任务. (3) 采用更大的 Batch Size 和词典大小. 本实验中使用的 RoBERTa-wwm-ext 则是针对中文数据的改进, 主要包括: (1) 采用中文整词掩码而非单个汉字. 例如, “牛子” 作为一个整体被掩蔽, 得到 “中科大真是牛校出[MASK][MASK]”. (2) 采用来自百科、新闻、问答网站的扩展数据集, 相较于中文维基百科, 语料规模增加了 10 倍以上. 我们期望 RoBERTa-wwm-ext 在中文文本分类任务上取得更好的效果.

\subsection{两种微调方式: 冻结与不冻结}

视觉基础模型的微调过程通常只对预测头进行参数更新, 而预训练权重一律冻结 (例如, 第一次作业). 然而, 基于 Transformer 的预训练语言模型的微调往往对两者同时更新. 我们对两种情况均感兴趣, 因此将 BERT-base 和 RoBERTa-wwm-ext 分别冻结和不冻结, 计划进行对比.

在我们的模型架构中, BERT 模型由预训练权重初始化, MLP 分类器则随机初始化, 按照同样的训练策略进行更新. 与预训练过程不同, 微调属于有监督学习, 旨在最小化分类器输出 Logits 与标签之间的交叉熵损失. 我们将在下一节给出与训练相关的超参数.

\section{模型测试实验设计}

\subsection{对比实验设计}

为了展示不同预训练权重和微调方式对模型性能的影响, 我们设计了四组实验, 对应如下四个模型:

\begin{enumerate}[noitemsep,topsep=0pt,parsep=0pt]
    \item \textbf{BERT-base (frozen)}: 加载 \texttt{bert-base-chinese} 权重, 词嵌入冻结, 仅分类器更新.
    \item \textbf{RoBERTa-wwm-ext (frozen)}: 加载 \texttt{hfl/chinese-roberta-wwm-ext} 权重, 词嵌入冻结, 仅分类器更新.
    \item \textbf{BERT-base}: 加载 \texttt{bert-base-chinese} 权重, 词嵌入和分类器同时更新.
    \item \textbf{RoBERTa-wwm-ext}: 加载 \texttt{hfl/chinese-roberta-wwm-ext} 权重, 词嵌入和分类器同时更新.
\end{enumerate}

对于每一个模型, 我们使用 Fold 2\textasciitilde 5 作为训练集, Fold 1 作为验证集. 将准确率和 $F_1$ 分数作为评价指标. 在训练过程中记录验证集上的损失函数及准确率变化. 最终, 我们将在测试集上比较文本二分类的准确率及 $F_1$ 分数.

\subsection{超参数设置}

预训练 BERT 模型的超参数与原论文保持一致, Transformer 编码器的模块数为 $L = 12$, 注意力头的数量为 $H = 12$, 词向量的特征维度为 $d = 768$, 使用 GELU 激活函数, 模型参数总量约为 1.1 亿. 此外, 我们自行设计 MLP 分类器, 隐层大小为 $2d = 1536$, 输出层大小为 2, 采用 SiLU 激活函数, 交叉熵损失函数, 不使用标签平滑. 为了防止过拟合, 我们在 MLP 分类器中采用了 0.1 的 Dropout 概率. 优化器采用 Adam, 学习率为 $2\times 10^{-5}$, 批大小为 32, 训练 3 个 Epoch. 以上超参数设置对所有四个模型适用. 模型架构的更多细节参见实验代码.

本实验所有模型采用 Tensorflow 和 Keras 实现, 在 1 块 NVIDIA GeForce GTX 1080 Ti GPU 上完成微调.

\section{模型测试结果及分析}

实验主要结果如表 \ref{tab:confusion}, \ref{tab:results} 和图 \ref{fig:learning-curves} 所示.

\subsection{混淆矩阵}


\begin{table}[htbp]
    \centering\small
    \subcaptionbox{BERT-base (frozen)}[0.45\textwidth]{
        \begin{tblr}{width={0.3\textwidth},colspec={c*{3}{X[c]}},hline{3-4}={2-4}{solid},vline{3-4}={2-4}{solid}}
                                                               &   & \SetCell[c=2]{m}{Predicted} &     \\
                                                               &   & 0                           & 1   \\
            \SetCell[r=2]{m}{\rotatebox[origin=c]{90}{Actual}} & 0 & 187                         & 12  \\
                                                               & 1 & 6                           & 196 \\
        \end{tblr}
    }
    \subcaptionbox{RoBERTa-wwm-ext (frozen)}[0.45\textwidth]{
        \begin{tblr}{width={0.3\textwidth},colspec={c*{3}{X[c]}},hline{3-4}={2-4}{solid},vline{3-4}={2-4}{solid}}
                                                               &   & \SetCell[c=2]{m}{Predicted} &     \\
                                                               &   & 0                           & 1   \\
            \SetCell[r=2]{m}{\rotatebox[origin=c]{90}{Actual}} & 0 & 186                         & 13  \\
                                                               & 1 & 8                           & 194 \\
        \end{tblr}
    }\\
    \subcaptionbox{BERT-base}[0.45\textwidth]{
        \begin{tblr}{width={0.3\textwidth},colspec={c*{3}{X[c]}},hline{3-4}={2-4}{solid},vline{3-4}={2-4}{solid}}
                                                               &   & \SetCell[c=2]{m}{Predicted} &     \\
                                                               &   & 0                           & 1   \\
            \SetCell[r=2]{m}{\rotatebox[origin=c]{90}{Actual}} & 0 & 196                         & 3   \\
                                                               & 1 & 4                           & 198 \\
        \end{tblr}
    }
    \subcaptionbox{RoBERTa-wwm-ext}[0.45\textwidth]{
        \begin{tblr}{width={0.3\textwidth},colspec={c*{3}{X[c]}},hline{3-4}={2-4}{solid},vline{3-4}={2-4}{solid}}
                                                               &   & \SetCell[c=2]{m}{Predicted} &     \\
                                                               &   & 0                           & 1   \\
            \SetCell[r=2]{m}{\rotatebox[origin=c]{90}{Actual}} & 0 & 198                         & 1   \\
                                                               & 1 & 4                           & 198 \\
        \end{tblr}
    }
    \caption{不同版本模型的混淆矩阵}
    \label{tab:confusion}
\end{table}

表 \ref{tab:confusion} 给出了四个版本模型在测试集上的混淆矩阵. 可见, 四个模型均能准确分类大部分中文文本样本. 其中 RoBERTa-wwm-ext 的结果最优, 仅有 1 个标签为“0”的样本被误分类为“1”, 4 个标签为“1”的样本被误分类为“0”.

\begin{table}[htbp]
    \centering\small
    \begin{tblr}{ccl}
        \toprule
        Actual & Predicted & Text                \\
        \midrule
        0      & 1         & 美媒:“海归派”在中国势力逐渐壮大   \\
        1      & 0         & 武汉高校公寓楼突发大火 百余人获疏散  \\
        1      & 0         & 我国10个专业学位硕士将增招应届毕业生 \\
        1      & 0         & 教育部:加大对民族教育支持力度     \\
        1      & 0         & 高校毕业生将可获至少10天创业培训   \\
        \bottomrule
    \end{tblr}
    \caption{RoBERTa-wwm-ext 错误分类的样本}
\end{table}

我们列举出一些错误分类的样本, 并尝试分析其原因. 观察发现, 被错误分类的“0”与高等教育话题相关, 可能是由于涉及“美媒”“中国”等字眼, 因此模型错误预测为“1”. 被错误分类的“1”同时与教育和国家政策相关, 标签本身存在一定的噪声或混淆, 因此导致模型错误分类. 总体而言, 微调得到的模型能够合理地区分两类文本.

\subsection{性能比较}


\begin{table}[htbp]
    \centering\small
    \begin{tblr}{width={0.8\textwidth},colspec={l|X*{2}{S[table-format=1.4]}}}
        \toprule
        Dataset                  & Model                    & {{{Accuracy}}} & {{{$F_1$ Score}}} \\
        \midrule
        \SetCell[r=4]{l}{Train.} & BERT-base (frozen)       & 0.9492         & 0.9492            \\
                                 & RoBERTa-wwm-ext (frozen) & 0.9515         & 0.9515            \\
                                 & BERT-base                & 0.9906         & 0.9906            \\
                                 & RoBERTa-wwm-ext          & 0.9898         & 0.9898            \\
        \midrule
        \SetCell[r=4]{l}{Valid.} & BERT-base (frozen)       & 0.9563         & 0.9562            \\
                                 & RoBERTa-wwm-ext (frozen) & 0.9625         & 0.9625            \\
                                 & BERT-base                & 0.9781         & 0.9781            \\
                                 & RoBERTa-wwm-ext          & 0.9781         & 0.9781            \\
        \midrule
        \SetCell[r=4]{l}{Test}   & BERT-base (frozen)       & 0.9551         & 0.9551            \\
                                 & RoBERTa-wwm-ext (frozen) & 0.9476         & 0.9476            \\
                                 & BERT-base                & 0.9825         & 0.9825            \\
                                 & RoBERTa-wwm-ext          & 0.9875         & 0.9875            \\
        \bottomrule
    \end{tblr}
    \caption{不同版本模型的分类准确率及 $F_1$ 分数}
    \label{tab:results}
\end{table}

表 \ref{tab:results} 给出了四个版本模型在训练集、验证集和测试集上的分类准确率及 $F_1$ 分数. 我们可以看到, 四种模型均能较好地区分两类文本, 准确率普遍达到 95\% 乃至更高, 体现出基于 BERT 的预训练-微调范式的优越性.

此外, RoBERTa-wwm-ext 的性能略优于 BERT-base, 说明 RoBERTa-wwm-ext 在中文文本分类任务上的确取得了更好的效果, 测试集准确率高达 98.75\%, 体现出整词掩蔽、扩展数据集等预训练改进的有效性.

不同的微调方式也会对模型性能产生较大的影响. 在 BERT-base 上, 词嵌入与分类器共同更新使得准确率相较冻结的情况提高了 2.74 个百分点, 而在 RoBERTa-wwm-ext 上提高了 3.99 个百分点. 这说明词嵌入与下游任务的相关性较强, 适合与分类器共同更新, 以达到域适应的效果.

\subsection{训练曲线分析}

\begin{figure}[htbp]
    \centering
    \subfloat[BERT 冻结的情况下]{
        \includegraphics[width=0.95\textwidth]{./figs/bert-curves-frozen.pdf}
    }\\
    \subfloat[BERT 不冻结的情况下]{
        \includegraphics[width=0.95\textwidth]{./figs/bert-curves.pdf}
    }
    \caption{训练过程中的损失函数及准确率变化}
    \label{fig:learning-curves}
\end{figure}


图 \ref{fig:learning-curves} 给出了四个版本模型在训练过程中的损失函数及准确率变化. 冻结词嵌入进行微调具有更小的计算量和更快的训练速度, 但是在相同 Epoch 的情况下收敛水平远不及同时更新词嵌入和分类器的模型. 这进一步证明了微调方式对模型性能的重要性.

此外, 针对 RoBERTa-wwm-ext 在两种情况下均取得比 BERT-base 更好的收敛水平, 与我们的预期相符, 体现了先进的预训练策略的有效性.

\section{实验总结}

本实验中, 我们基于 BERT 和 RoBERTa, 采用预训练-微调范式, 设计了中文文本二分类模型, 在中文文本分类任务上取得了较好的效果. 通过对比实验, 我们发现: (1) 相较 BERT-base, 针对中文改进了预训练策略的 RoBERTa-wwm-ext 在中文文本分类任务上取得了更好的效果; (2) 相较单独更新分类器, 词嵌入与分类器共同更新的方式能够进一步提升模型性能, 适合于 BERT 等预训练模型.

\section{参考文献}

\renewcommand{\bibsection}{}
\bibliographystyle{unsrt}
\bibliography{refs}

\end{document}
