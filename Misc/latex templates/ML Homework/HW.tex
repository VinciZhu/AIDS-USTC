\documentclass[11pt,notitlepage]{article}
\usepackage{fancyhdr}
\usepackage{fancybox}
\usepackage{graphicx}
\usepackage{pdfpages}
\usepackage{enumitem}
\usepackage{algorithm}
\usepackage[noend]{algorithmic}
\renewcommand{\headrulewidth}{1.5pt}
\renewcommand{\footrulewidth}{1.5pt}
\usepackage[framemethod=TikZ]{mdframed}
\usepackage{amssymb,amsmath}
\usepackage{amsthm}
\usepackage{bm, mathtools}
\usepackage{thmtools}
\usepackage{caption,subcaption}
\usepackage[breakable,skins]{tcolorbox}
\RequirePackage{ifthen,currfile-abspath}
\getabspath{\jobname.log}
\ifthenelse{\equal{\theabsdir}{\thepwd}}
{}{\PassOptionsToPackage{outputdir=\theabsdir}{minted}}
\usepackage{minted}
\setminted{autogobble,breaklines,fontsize=\small}
\usepackage{booktabs,multirow}
% \usemintedstyle{bw}
\setlength{\topmargin}{-2pt}
\setlength{\textheight}{9in}
\setlength{\headheight}{13.59999pt}
\setlength{\oddsidemargin}{0in}
\setlength{\textwidth}{6.5in}
\fancypagestyle{plain}{
	\fancyhf{}
	\cfoot{\thepage}
	\renewcommand{\headrulewidth}{0pt}
	\renewcommand{\footrulewidth}{1.5pt}
}
\pagestyle{fancy}
%%%%%%%%%%%%%%%%%%%%%%%
%% Define the Exercise environment %%
%%%%%%%%%%%%%%%%%%%%%%%%
\mdtheorem[
	topline=false,
	rightline=false,
	leftline=false,
	bottomline=false,
	leftmargin=-10,
	rightmargin=-10
]{exercise}{Exercise}
%%%%%%%%%%%%%%%%%%%%%%%
%% End of the Exercise environment %%
%%%%%%%%%%%%%%%%%%%%%%%
%%%%%%%%%%%%%%%%%%%%%%%
%% Define the Problem environment %%
%%%%%%%%%%%%%%%%%%%%%%%
\mdtheorem[
	topline=false,
	rightline=false,
	leftline=false,
	bottomline=false,
	leftmargin=-10,
	rightmargin=-10
]{problem}{Problem}
%%%%%%%%%%%%%%%%%%%%%%%
%% End of the Exercise environment %%
%%%%%%%%%%%%%%%%%%%%%%%
%%%%%%%%%%%%%%%%%%%%%%%
%% Define the Solution Environment %%
%%%%%%%%%%%%%%%%%%%%%%%
\declaretheoremstyle[
	spaceabove=\parsep,
	spacebelow=\parsep,
	headfont=\normalfont\bfseries,
	notefont=\mdseries,
	notebraces={(}{)},
	headpunct={: \\},
	headindent={},
	postheadspace=4pt,
	bodyfont=\normalfont,
	qed=$\blacksquare$,
]{mystyle}
\declaretheorem[style=mystyle,title=Solution,numbered=no]{solution}
% \tcolorboxenvironment{solution}{ 
%     breakable, blank, borderline={0.5pt}{0pt}{black},
% }
%%%%%%%%%%%%%%%%%%%%%%%
%% End of the Solution environment
%%%%%%%%%%%%%%%%%%%%%%%

\renewcommand{\eqref}[1]{Eq.~(\ref{#1})}
\newcommand{\figref}[1]{Fig.~\ref{#1}}


\newcommand{\proj}[2]{\Pi_{#2} \left(#1\right)}
\newcommand{\rank}[1]{ \operatorname{rank} \left(#1\right) }
\newcommand{\tr}{ \operatorname{tr} }
\newcommand{\diag}{ \operatorname{diag} }
\newcommand{\adj}{ \operatorname{adj} }
\newcommand{\cof}{ \operatorname{cof} }
\newcommand{\sgn}{ \operatorname{sgn} }
\newcommand{\Var}[1]{ \operatorname{Var} \left(#1\right) }
\newcommand{\Cov}[1]{ \operatorname{Cov} \left(#1\right) }
\newcommand{\amin}{ \mathop{\arg\min} }
\newcommand{\amax}{ \mathop{\arg\max} }
\newcommand{\prox}{ \operatorname{prox} }
\newcommand{\softmax}{ \operatorname{softmax} }
\newcommand{\sigmoid}{ \operatorname{sigmoid} }
\newcommand{\logit}{ \operatorname{logit} }

\newcommand{\mb}[1]{\mathbf{#1}}
\newcommand{\mc}[1]{\mathcal{#1}}
\newcommand{\R}{\mathbb{R}}
\newcommand{\diff}{\text{d}}

\DeclareMathOperator*{\argmin}{\bf argmin\,}
\DeclareMathOperator*{\argmax}{\bf argmax\,}
\DeclareMathOperator*{\relint}{\bf relint\,}
\DeclareMathOperator*{\relbd}{\bf relbd\,}
\DeclareMathOperator*{\dom}{\bf dom\,}
\DeclareMathOperator*{\intp}{\bf int\,}
\DeclareMathOperator*{\cl}{\bf cl\,}
\DeclareMathOperator*{\bd}{\bf bd\,}
\DeclareMathOperator*{\lspan}{\bf span\,}
\DeclareMathOperator*{\aff}{\bf aff\,}
\DeclareMathOperator*{\cone}{\bf cone\,}
\DeclareMathOperator*{\conv}{\bf conv\,}
\DeclareMathOperator*{\epi}{\bf epi\,}

% Definition environment	
\theoremstyle{definition}
\newtheorem{definition}{Definition}

% head
\lhead{\textbf{\name}}
\rhead{\textbf{\id}}
\chead{\textbf{Homework \hwno}}

% title
\renewcommand{\maketitle}{
	\vspace*{-4\baselineskip}
	\thispagestyle{plain}
	\begin{center}
		{\bf\large \course}\\
		\semester\\
		\school
	\end{center}
	\noindent
	Lecturer: \lecturer\hfill
	Homework \hwno\\
	Name: \name\hfill
	ID: \id\\
	\noindent
	\rule{\textwidth}{2pt}
	\medskip
	\textbf{Notice, }to get the full credits, please present your solutions step by step.
}